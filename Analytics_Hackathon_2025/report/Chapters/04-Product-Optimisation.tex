\chapter[Product Optimization]{Product Optimization}
\label{po:product-optimization}

\section{Introduction}
This chapter evaluates product performance to identify opportunities for portfolio optimization, improved profitability, and strategic cross-selling. The analysis focuses on four goals: identifying profitability extremes (hero vs. laggard products), analyzing cross-selling patterns, evaluating channel-specific trends, and assessing discount sensitivity.

\section{Methodology}
The analysis relied on sales and product data to assess profitability, bundling opportunities, and discount impacts. Methodologies included:
\begin{itemize}
	\item \textbf{Profitability Analysis:} Products were ranked by revenue and profit margin to identify top and bottom performers.
	\item \textbf{Market Basket Analysis:} A self-join on sales data was used to find frequently co-purchased product pairs, revealing bundling opportunities.
	\item \textbf{Discount Impact Analysis:} Sales were segmented into discount buckets (e.g., 0-5\%, >30\%) to measure the effect of discounts on sales volume and profitability for different product categories.
	\item \textbf{Channel-Specific Analysis:} Product category performance was compared across the Online and Reseller channels to identify distinct sales patterns.
\end{itemize}

\section{Key Findings}

\subsection{Hero vs. Laggard Products}
The product portfolio exhibits significant performance variance. The \textit{Mountain-200} series stands out as a "hero" product line, consistently delivering high revenue and strong profit margins (15-20\%). Conversely, the \textit{Road-250 Black} series, despite being a top-10 revenue generator, is a "laggard" product that operates at a negative profit margin, effectively losing money on each sale.

\begin{figure}[!htpb]
	\centering
	\includegraphics[width=0.8\linewidth]{Figures/fig_5.png}
	\caption{Revenue vs. profit margin for top 10 products.}
	\label{fig:revenue_margin_top10}
\end{figure}

\subsection{Market Basket and Bundling Opportunities}
Market basket analysis revealed strong cross-selling patterns that suggest clear bundling opportunities. Key patterns include:
\begin{itemize}
	\item \textbf{Component Pairing:} High-frequency pairing of components like frames and forks, common in reseller assembly orders.
	\item \textbf{Accessory Add-ons:} Frequent co-purchase of accessories, such as helmets and patch kits, particularly in the Online channel.
	\item \textbf{Upsell Paths:} A significant number of customers purchasing both \textit{Mountain-200} and \textit{Mountain-300} series bikes, indicating a clear upsell opportunity.
\end{itemize}

\begin{table}[h!]
	\centering
	\caption{Some frequently purchased product pairs.}
	\label{tab:product_pairs}
	\begin{tabularx}{\textwidth}{l l X X}
		\toprule
		\textbf{Product A}      & \textbf{Product B}      & \textbf{Frequency} & \textbf{Insight}                 \\
		\midrule
		LL Road Frame - Red, 48 & Road-650 Black, 58      & 225                & High-volume component pairing.   \\
		Road-650 Black, 58      & Road-650 Black, 60      & 319                & Bulk reseller stocking.          \\
		Sport-100 Helmet, Red   & Patch Kit/8 Patches     & 132                & Strong accessory add-on.         \\
		Mountain-200 Silver, 38 & Mountain-300 Black, 40  & 193                & Cross-model buyer overlap.       \\
		Hitch Rack - 4-Bike     & Touring-3000 Yellow, 54 & 71                 & High-value accessory attachment. \\
		\bottomrule
	\end{tabularx}
\end{table}

\subsection{Channel-Specific Product Performance}
Product categories perform very differently depending on the sales channel. \textbf{Components} are sold exclusively through the Reseller channel, generating \$11.8M in revenue with zero online presence. In contrast, \textbf{Accessories} are most popular online, with 55\% of their revenue coming from that channel. \textbf{Bikes} remain heavily reliant on the Reseller channel, which accounts for 70\% of bike revenue.

\begin{figure}[!htpb]
	\centering
	\includegraphics[width=0.8\linewidth]{Figures/fig_6.png}
	\caption{Category revenue mix by channel.}
	\label{fig:category_revenue_channel}
\end{figure}

\subsection{The Corrosive Impact of Aggressive Discounting}
Discounting has a dramatically different impact across product categories. For Road Bikes, minimal discounts (0-5\%) yield a healthy profit of \$62 per unit. However, deep discounts (>30\%) on Mountain Bikes lead to a loss of \textbf{\$846 per unit}. This demonstrates that undisciplined discounting, likely used to drive volume in the Reseller channel, is a primary driver of unprofitability.

\begin{figure}[!htpb]
	\centering
	\begin{subfigure}{0.45\textwidth}
		\centering
		\includegraphics[width=\textwidth]{Figures/fig_8.png}
		\caption{Discounting influencing profit in the Road Bikes category.}
		\label{fig:discount_volume}
	\end{subfigure}
	\hspace{.5cm}
	\begin{subfigure}{0.45\textwidth}
		\centering
		\includegraphics[width=\textwidth]{Figures/fig_9.png}
		\caption{Discounting influencing losses in the Mountain Bikes category.}
		\label{fig:discount_margin}
	\end{subfigure}
	\caption{The impact of discounting on profitability across product categories.}
	\label{fig:discount_analysis}
\end{figure}

\section{Business Insights and Recommendations}
The product portfolio is unbalanced, with a handful of hero products subsidizing a long tail of underperformers and outright loss-makers. Channel-specific product strategies and disciplined pricing are essential for improving overall profitability.

\begin{enumerate}
	\item \textbf{Conduct a profitability audit on the Road-250 series:} The \textit{Road-250 Black} series is a major revenue driver but a net loss for the company. An immediate review of its cost of goods sold (COGS), pricing, and promotional strategy is required. The business must decide whether to adjust pricing to make it profitable or discontinue it.
	\item \textbf{Implement an improved discounting policy:} Introduce a company-wide policy that caps discounts, particularly on high-value items like Mountain Bikes and Road Bikes. A 15\% discount ceiling should be enforced, with any exceptions requiring senior management approval. This would immediately address the significant losses from deep discounting.
	\item \textbf{Create strategic product bundles:} Capitalize on the market basket analysis by creating official product bundles. For example, a "Reseller Build Kit" could package popular frame and component pairs, while an "Online Adventure Bundle" could combine a bike with high-margin accessories like helmets and lights.
	\item \textbf{Develop channel-specific product strategies:} Acknowledge and formalize the de facto channel specializations. Position the Reseller channel as the exclusive home for components and complex bike builds. Position the Online channel as the primary destination for accessories, clothing, and direct-to-consumer bike sales with a focus on higher-margin configurations.
\end{enumerate}

% \section{Limitations}
% \begin{itemize}
% 	\item \textbf{Cost Data Granularity:} The analysis is based on standard cost data, which may not capture all manufacturing and supply chain costs. The root cause of high COGS in laggard products requires a more detailed financial investigation.
% 	\item \textbf{Temporal Blind Spots:} The aggregate analysis does not account for product lifecycle stage (e.g., launch vs. decline) or seasonality, which could influence profitability.
% 	\item \textbf{Lack of Competitive Context:} Without market pricing data, it is difficult to determine whether losses on certain products are due to competitive pressures or internal inefficiencies.
% \end{itemize}
