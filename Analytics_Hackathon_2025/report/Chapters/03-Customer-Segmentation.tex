\chapter[Customer Segmentation]{Customer Segmentation Analysis}
\label{cp:customer-segmentation}

\section{Introduction}
Following the analysis of channel performance, this chapter delves into the customer base to identify distinct segments with similar purchasing behaviors. By grouping customers, AdventureWorks can tailor marketing strategies, optimize resource allocation, and enhance customer lifetime value (CLV). The analysis focuses exclusively on the Online channel to understand direct-to-consumer behavior.

\section{Methodology}
The primary method employed was \textbf{RFM Analysis (Recency, Frequency, Monetary)}, a rule-based segmentation technique chosen for its interpretability and direct applicability to marketing actions.

\vspace{1em}
\textbf{Data Preparation:}
The analysis used sales transaction data from the Online channel, joined with customer demographic and geographic information. Key data sources included aggregated transactional data, geographic distributions, and repeat purchase metrics. Preprocessing involved:
\begin{enumerate}
    \item \textbf{Filtering:} Restricting the dataset to Online channel sales.
    \item \textbf{Aggregation:} Grouping transactions by \texttt{CustomerID} to calculate RFM metrics.
    \item \textbf{Feature Engineering:} Deriving \texttt{Recency} (days since last purchase), \texttt{Frequency} (total number of orders), and \texttt{Monetary} (total spend).
\end{enumerate}

\textbf{Scoring Logic:}
Customers were scored on a scale of 1 to 4 for each RFM dimension using the \texttt{NTILE(4)} window function:
\begin{itemize}
    \item \textbf{Recency (R):} Higher scores for more recent purchases.
    \item \textbf{Frequency (F):} Higher scores for more frequent purchases.
    \item \textbf{Monetary (M):} Higher scores for higher total spend.
\end{itemize}
An aggregate \texttt{RFM\_Score} (R + F + M) was used to classify customers into five segments: \textit{Champions}, \textit{Loyal Customers}, \textit{Potential Loyalists}, \textit{At-Risk Customers}, and \textit{Lost Customers}.

\section{Key Findings}

\subsection{Segment Distribution}
The analysis of 18,484 online customers revealed a highly concentrated value structure. As shown in \autoref{tab:segment_distribution}, the \textbf{Champions} (5.5\% of customers) and \textbf{Loyal Customers} (33.8\%) represent the high-value core of the business. Conversely, the \textbf{At-Risk} and \textbf{Lost} segments make up over 20\% of the customer base, representing a significant pool of disengaged accounts.

\begin{table}[h!]
    \centering
    \caption{Customer Segmentation Distribution}
    \label{tab:segment_distribution}
    \begin{tabularx}{\textwidth}{l X r r}
        \toprule
        \textbf{Segment}    & \textbf{Customer Count} & \textbf{\% of Base} & \textbf{Business Implication}          \\
        \midrule
        Champions           & 1,007                   & 5.5\%               & High-value core; maximize retention.   \\
        Loyal Customers     & 6,257                   & 33.8\%              & Stable revenue base; nurture loyalty.  \\
        Potential Loyalists & 7,319                   & 39.6\%              & Largest segment; prime growth target.  \\
        At-Risk Customers   & 2,907                   & 15.7\%              & Require win-back; high recovery value. \\
        Lost Customers      & 994                     & 5.4\%               & Low-touch or reactivation pilots only. \\
        \midrule
        \textbf{Total}      & \textbf{18,484}         & \textbf{100\%}      & ---                                    \\
    \end{tabularx}
\end{table}

\begin{figure}[!htpb]
    \centering
    \includegraphics[width=0.5\linewidth]{Figures/fig_3.png}
    \caption{Customer count distribution across RFM segments.}
    \label{fig:segment-distribution}
\end{figure}

\subsection{The 161-Day Repeat Purchase Cycle}
A key behavioral insight is that the average time to a second purchase is \textbf{161 days} (approximately 5.3 months). This metric provides a clear window of opportunity for targeted marketing interventions to encourage repeat business and prevent customer churn.

\subsection{New vs. Repeat Customer Value}
Repeat customers are significantly more valuable than new customers, with Average Order Values (AOV) that are \textbf{8 to 20 times higher}. New customers typically have an AOV around \$1,600, while repeat customers spend \$6,000 or more per order. This highlights the strategic importance of retention over acquisition.

\begin{figure}[!htpb]
    \centering
    \includegraphics[width=\linewidth]{Figures/fig_4.png}
    \caption{Average order value comparison: new versus repeat customers by month.}
    \label{fig:aov-new-vs-repeat}
\end{figure}

\subsection{Geographic Concentration of Top Customers}
The top 10\% of customers are highly concentrated geographically. \textbf{England} (66 top customers) and the \textbf{Paris} region (33 top customers) are the most valuable international clusters. Australia also represents a significant base, with 52 top customers across Queensland and New South Wales.

\section{Business Insights and Recommendations}
The segmentation analysis provides a clear roadmap for data-driven marketing and retention strategies. The extreme concentration of value in a small customer segment, combined with the predictable 161-day repeat purchase cycle, offers powerful levers for growth.

\begin{enumerate}
    \item \textbf{Implement a Timed Nurture Campaign:} Launch an automated email and retargeting campaign that triggers between 90 and 120 days after a customer's first purchase. The goal is to re-engage customers during their consideration phase, just before the 161-day average repeat purchase window closes.
    \item \textbf{Develop a "Champion" VIP Program:} Create an exclusive VIP program for the "Champion" segment. Benefits could include early access to new products, a dedicated customer service line, and invitations to special events. This focuses resources on retaining the most profitable customer group.
    \item \textbf{Launch a "Potential to Loyal" Conversion Initiative:} Target the largest segment, "Potential Loyalists," with campaigns designed to increase their purchase frequency. Offer a small incentive (e.g., 15\% off) for making a second purchase within 90 days to shorten the natural repeat cycle.
    \item \textbf{Focus Geographic Marketing Efforts:} Allocate marketing and potentially logistical resources to the top geographic clusters: England, Paris, and Australia. This could include localized marketing content, targeted digital advertising, and exploring regional fulfillment options to improve service.
\end{enumerate}

\section{Limitations}
\begin{itemize}
    \item \textbf{Rank-Based Scoring:} The NTILE quartile system can mask variance within segments. For example, a customer with a \$50k lifetime spend might be in the same monetary quartile as one with a \$5k spend.
    \item \textbf{Data Recency:} Recency calculations are tied to the date the analysis was run and may not reflect current customer status.
    \item \textbf{Temporal Bias:} Newer customer cohorts have had less time to make repeat purchases, which may skew retention metrics downward compared to older cohorts.
    \item \textbf{Geographic Granularity:} The data is at the state/province level, which prevents more granular, city-level micro-targeting.
\end{itemize}
