\chapter[Channel Performance and Salesforce Efficiency]{Channel Performance Analysis}
\label{cp:channel-performance}

{
	\parindent0pt
	\vspace{.935em}
	\section{Overview of Sales Channels}

	AdventureWorks distributes its products through two channels: the \textbf{Online} channel (Direct-to-Consumer) and the \textbf{Reseller} channel (Business-to-Business). The Reseller channel is responsible for the highest sales volume, but the Online channel is the most stable source of profit.

	\section{Financial Performance and The Profitability Paradox}

	A key finding from the longitudinal data (2011–2014) is that channels with higher revenue volume generally had lower profit margins.

	\vspace{1em}
	As illustrated in \autoref{tab:profitability_paradox}, the Reseller channel consistently generates the majority of revenue, peaking at \$32.89 million in 2013. However, this volume appears to come at a cost. The Reseller channel has operated at a negative profit margin since 2012, recording a loss of -2.85\% in 2013. Conversely, the Online channel, while smaller in scale, maintains a profit margin of about 40\%.

	\begin{table}[h!]
		\centering
		\caption{Yearly Revenue and Profit Margin Comparison}
		\label{tab:profitability_paradox}
		\begin{tabularx}{\textwidth}{l X r r r}
			\toprule
			\textbf{Year}          & \textbf{Channel} & \textbf{Total Revenue (\$)} & \textbf{Total Profit (\$)} & \textbf{Margin (\%)} \\
			\midrule
			\multirow{2}{*}{2011}  & Online           & 3.86 M                      & 1.54 M                     & 39.91\%              \\
			                       & Reseller         & 8.78 M                      & 0.08 M                     & 0.97\%               \\
			\midrule
			\multirow{2}{*}{2012}  & Online           & 6.39 M                      & 2.38 M                     & 37.28\%              \\
			                       & Reseller         & 27.13 M                     & (1.43 M)                   & -5.29\%              \\
			\midrule
			\multirow{2}{*}{2013}  & Online           & 10.73 M                     & 4.29 M                     & 40.00\%              \\
			                       & Reseller         & 32.89 M                     & (0.94 M)                   & -2.85\%              \\
			\midrule
			\multirow{2}{*}{2014*} & Online           & 8.37 M                      & 3.47 M                     & 41.45\%              \\
			                       & Reseller         & 11.69 M                     & (0.03 M)                   & -0.24\%              \\
			\bottomrule
			\multicolumn{5}{l}{\footnotesize \textit{*2014 data represents a partial fiscal year.}}
		\end{tabularx}
	\end{table}

	This discrepancy suggests that the pricing strategy for Resellers—likely involving deep volume discounts—may be eroding the bottom line, whereas the Online channel benefits from premium pricing power.

	\section{Operational Metrics: Volume vs. Value}

    Average Order Value (AOV) tells you how much money, on average, each customer spends per transaction.
   

	\begin{itemize}
		\item \textbf{Reseller Channel:} High AOV of \textbf{\$21,147.58} and an average of \textbf{16 items per order}. This indicates complex, large-scale logistical requirements.
		
		\item \textbf{Online Channel:} Lower AOV of \textbf{\$1,061.45} and an average of \textbf{2.18 items per order}, reflecting a transactional, high-frequency retail model.
	\end{itemize}

     This confirms that the Reseller channel functions as a bulk-wholesale mechanism, while the Online channel serves individual consumer needs.

	\section{Geographic \& Personnel Performance}

	\subsection{Territorial Growth}
	The North American market remains the stronghold for AdventureWorks. The \textbf{Southwest} territory, for instance, is a dominant performer, generating over \$7.19 million in 2013 alone as seen in \autoref{fig:yoy-growth}. However, international markets are showing promise. The \textbf{United Kingdom} and \textbf{France} showed positive year-over-year growth in 2013 (1.23\% and 1.61\% respectively), identifying them as key expansion targets. On the other hand, established markets like Central North America have shown signs of saturation or decline (-0.68\% growth in 2014).

        \begin{figure}[!htpb]
            \centering
            \includegraphics[width=\linewidth]{Figures/fig_1.png}
            \caption{Regional revenue distribution across territories (all years).}
            \label{fig:territorial-revenue}
        \end{figure}

        \begin{figure}[!htpb]
            \centering
            \includegraphics[width=\linewidth]{Figures/fig_2.png}
            \caption{Trend of revenue growth by territory.}
            \label{fig:yoy-growth}
        \end{figure}

        \vspace{1em}
	\subsection{Salesforce Efficiency}
	An analysis of the top salespeople within the Reseller channel reiterates the concern regarding profitability noted earlier. \autoref{tab:salesperson_performance} lists the top five performers by revenue. Notably, \textbf{all top five salespeople generated negative profit margins}.
	\begin{table}[h!]
		\centering
		\caption{Top 5 Salespeople by Revenue (Reseller Channel)}
		\label{tab:salesperson_performance}
		\begin{tabularx}{\textwidth}{l X r r r}
			\toprule
			\textbf{Salesperson} & \textbf{} & \textbf{Total Revenue} & \textbf{Total Profit} & \textbf{Profit Margin} \\
			\midrule
			Michael Blythe & & \$9,293,903 & (\$281,662) & -3.03\% \\
			Jae Pak & & \$8,503,338 & (\$142,034) & -1.67\% \\
			Tsvi Reiter & & \$7,171,012 & (\$147,095) & -2.05\% \\
			Shu Ito & & \$6,427,005 & (\$396,323) & -6.17\% \\
			Amy Alberts & & \$732,759 & (\$24,279) & -3.31\% \\
			\bottomrule
		\end{tabularx}
	\end{table}

	For example, Michael Blythe generated the highest revenue (\$9.29M) but incurred a loss of over \$281k. This suggests that sales commissions or targets may be aligned solely with gross revenue rather than net profitability, incentivizing aggressive discounting to close deals.


}
